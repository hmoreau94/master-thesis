\chapter{Conclusion}
This thesis thoroughly presents the empirical protocol that was followed in order to build a proof of concept in the scope of customer premises equipment failure detection.

We started by presenting the setup of the project and the technical background necessary to understand the work that has been done. This also allowed us to define the different resources available and derive the problem that was identified.

We moved to the description of the dataset construction. This phase certainly represented the most important of all. It consisted in gathering the different data sources available, selecting particular pieces of information that could prove to be useful for classification and finally determining how to best represent such information. The practical approach of data collection was also described as it required some engineering tricks to optimise our use of available resources.

Data analysis and machine learning was described in a trial and error spirit to justify our choices. The particular setup and goal of the project prevented us from using a classical pipeline. Clustering of the data to perform labelling propagation was attempted without much success but supported choices that could be later applied to the machine learning pipeline. 

Finally, we discussed our approach and tried to give recommendations regarding how we could use this proof of concept to take the project forward and implement it to increase the value to the customer and the company. 

To put things back in perspective, this \acrshort{poc} has drastically improved the set of tools that can be used by management to push for the development of internal competencies regarding Data Science. Out of the average of 0.2\% failing \acrshort{cpe}s, we have shown that we could detect 25\% of those with an 80\% accuracy proving the existence of an underlying pattern in the data that could be refined in order to proactively troubleshoot such failures. While this recall could seem low, it fits with the spirit of the project but also with the limited ressources that can be deployed to handle predictions at this stage. The project has sparked a lot of traction in the company both from the network and customer care business lines. 

Hopefully the desire of UPC to move to such analytical field will be confirmed in the future and leverage the research that has been conducted during this thesis to build an innovative solution.